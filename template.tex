\documentclass[9pt]{beamer}
%\usefonttheme{serif} 
\usepackage{multimedia}
\usepackage[scaled=.90]{helvet}
\usepackage{courier}
\usepackage[T1]{fontenc}
\usepackage{graphicx}
\usepackage{amsmath}
\usepackage{caption}
%This is a semiar template 
%created by Bijumon T
%Asst.Prof. In Computer Engg:
%Model Engg: College

\usetheme{JuanLesPins}
\begin{document}
\begin{frame}
%\flushleft{}
\title {\textbf{File Sharing Application using P2P Protocol}\\
}
\author { Emmanuel Antony (22) \\
Malavika S Menon (32)\\
Rose Joseph (48) \\
Varun Krishna S (61)\\ 
\hfill \break
Govt. Model Engineering College\\
 Thrikkakkara }\\
\maketitle
\end{frame}
\begin{frame}[shrink]{Outline}
\tableofcontents
\end{frame}

\section{Introduction}
\begin {frame}[shrink]{Introduction}
\begin{itemize}
\item At present, no major app provides a smooth user experience for transferring files using peer-to-peer protocols. 
\item We intend to develop a web application, with enhanced user experience for easier transfer of files based on P2P protocol.
\item  This would also help enhance speed and divide the payload borne by the sender for sharing the file to a large number of users. 
\end{itemize}
\end{frame}

\section{Areas of Interest}
\begin{frame} {Areas of Interest}

\begin{itemize}
	 \item Computer Networks
     \item Web \& App Development
\end{itemize}
\end {frame}

\section{Existing Systems }
\begin{frame} {Existing Systems }
The primary modes of sharing files that can be compared with the proposed solution are based on
\begin{itemize}
    \item WiFi Direct (Xender/ShareIt)
    \item Messaging Application(Whatsapp/Telegram)
    \item Cloud (Google Drive/Dropbox)
\end{itemize}
\end{frame} 

\subsection{WiFi Direct - Xender/ShareIt}
\begin{frame}{WiFi Direct - Xender/ShareIt}
 \begin{itemize}
     \item These Apps work using a technology called WiFi direct. It uses the WiFi card in your phone to connect with another device. 
     \item It sets up a private connection between the two devices so that you will be able to transfer files without any interference at very high speeds. 
     \item It is very similar to Bluetooth, one device acts as a server while the other is the client. The first step is creating a connection between the sockets of the two devices. 
     \item Once these dynamic sockets are linked you will be able to send data between them both.
 \end{itemize}
 \end{frame}

\subsection{Messaging Application - Whatsapp/Telegram/Signal}
\begin{frame}{Messaging Application}
 \begin{itemize}
     \item Traditional messaging applications like WhatsApp, Telegram and Signal can be used to share files. These apps upload the files onto a common server and the clients in the group can redownload it anytime they want.
     \item They go for a server client model. Works really well when the server is up and online.
     \item If the server is offline, like the Facebook downtime recently, using these services to share files is near impossible
     \item The exact mechanism by which the file is transferred differs based on the application
 \end{itemize}
                 
\end{frame}


\subsection{Cloud - Google Drive/Dropbox}
\begin{frame}{Cloud - Google Drive/Dropbox}
 \begin{itemize}
     \item Sharing files using cloud is a quick and easy process.
     \item Centralised servers are usually fast and upload and download speeds are quick.
     \item Like messaging applications, there is a single point of failure. And if cloud service provider is unavailable or if there is a server downtime, then using these services becomes near impossible.
    %  \item Google is closed source so we wouldn’t know how the files are stored, unless we manually encrypt it and upload it. There are privacy concerns that lie with it. 
 \end{itemize}
\end{frame}

\section{Disadvantages of Existing System}
\begin{frame}{Disadvantages of Existing System}
 \begin{itemize}
     \item Wifi Direct limited by capability of mobile hotspot. Uses direct point to point communication. 
     \item Cloud - Bandwidth wastage, one central server whose downtime will affect the entire network. 
     \item Messaging Applications - Focuses on a different area of messaging with file sharing, where as our application prioritises file sharing. 
 \end{itemize}
\end{frame}

\section{Why P2P?}
\begin{frame}{Why P2P?}
 \begin{itemize}
     \item Peer-to-peer is a model in which everyone becomes a server. There is no central server; everyone who uses the network acts as their own server. \item Speed on a traditional web host is quite limited. Especially for larger files, it would require a burst of speed that isn't sustainable for long periods and locks the server up for other users. Bandwidth is also costly. 
     \item In the client-server model, performance degrades with more users, as the same amount of bandwidth is shared among more people.
     \item In P2P, it's actually faster when more users download a file. Instead of taking the whole file from one user, you're taking smaller pieces from hundreds or thousands of others. Even if they only have a little bandwidth to spare, the combined connections mean you get the maximum speed possible. 
 \end{itemize}
    
\end{frame}


\section{Proposed System - Features }
\begin{frame}{Proposed System - Features }
 \begin{itemize}
     \item To use the existing torrent protocol as the base layer protocol and solve UI/UX pain points with respect to torrenting a file. 
     \item Have groups to which files can be shared easily p2p via the application instead of having to upload the asset to a centralised server.
     \item The application is meant to be web based primarily (likely to be over UDP).
     \item Once this is accomplished, protocol layer innovations will also be considered to make p2p file sharing more efficient for specific niche use cases.
 \end{itemize}
\end{frame}

\section{Literature Survey}
\begin{frame}{}
    \large Literature Survey
\end{frame}
\begin{frame}{P2P Networking: An Information-Sharing Alternative}
\begin{itemize}
     \item This paper considers P2P networks as an alternative to traditional-client server models & puts the various advantages and disadvantages it offers. 

\item Disadvantages
\begin{itemize}
   \item P2P networks use protocols that allow individual nodes to participate easily, but this could result in decreased security. 
\item Without a central server, each node may only contain a partial index of the members it knows.This results in poorer efficiency for search and retrieval. 
\item Listed information may be cluttered with significant noise.
\item The Wild Web - Letting any user freely share any type of information raises issues of copyright infringement, intellectual piracy, and the potential spread of undesirable content.

\item But, under a controlled environment, P2P networks can be effective for sharing.
\end{itemize}

\end{itemize}
\end{frame}

\begin{frame}{}
\begin{itemize}
    \item Advantages

\begin{itemize}
    \item Enhanced load balancing:
Several techniques used to monitor traffic and redistribute content between nodes if required. 
\item Dynamic Information Repository:
Any user can download information from a node and start distributing it too. So, content in high demand can be spread to other nodes. 
\item Redundancy and fault tolerance:
Replication of information at multiple nodes helps increase the availability. Decentralisation helps with fault tolerance. 
\item Content Based Addressing:
Addressing reaches a higher level in the semantic hierarchy because users specify a content identifier but not a physical location.
\end{itemize}
\end{itemize}
    
\end{frame}

\begin{frame}{PeerDB : A P2P-based System for Distributed Data Sharing}
\begin{itemize}
    \item Features 
    \begin{itemize}
    \item Each participating node is a full-fledged object management system that supports content based search.
\item Users can share data without a global schema
\item Adopts mobile agents to assist in query processing
\item Supports mechanisms to keep promising peers in close proximity based on some criterion.
\end{itemize}
\end{itemize}

\begin{itemize}
    \item Advantages
    \begin{itemize}
    \item Performs more than coarse level sharing of files
\item Better response time for queries 

\end{itemize}
\end{itemize}

\begin{itemize}
    \item Disadvantages
    \begin{itemize}
    \item Not primarily a file sharing application, mainly a distributed database management system.

\end{itemize}
\end{itemize}

\begin{itemize}
    \item Applications
    \begin{itemize}
    \item Health Care, Genomic Data, Data Caching

\end{itemize}
\end{itemize}
    
\end{frame}

\begin{frame}{The Bittorrent P2P File-Sharing System: Measurements and Analysis}
    \begin{itemize}
        \item BitTorrent is a P2P file-sharing prototype that has attracted millions of users. Makes up around 53\% of all P2P traffic.
        \item Only a file-downloading protocol. It uses other global components like websites for finding files.
        \item Advantages
        \begin{itemize}
            \item System of moderators help maintain the integrity of content on network and prevent pollution of data
\item Even average download speeds of 240 kbps allowed us to download large files in one day. 

        \end{itemize}
        \item Disadvantages
        \begin{itemize}
            \item Uses a global component like suprnova.org. When its server was down, it affected the entire network.
\item The system of moderators need global components, not distributed. 
\item When the popularity drops and the last peer/seed with certain content goes offline, the content dies. 
\item Peers should be given incentives to seed, such as bartering a file.

        \end{itemize}
    \end{itemize}
\end{frame}


\begin{frame}{Looking up data in P2P}
    \begin{itemize}
        \item The main challenge in P2P computing is to design and implement a robust and scalable distributed system composed of inexpensive, individually unreliable computers in unrelated administrative domains. The participants in a typical P2P system might include computers at homes, schools, and businesses, and can grow to several million concurrent participants.
        \item The ability to lookup data to access the data becomes very important in the process
        \item Advantages
        \begin{itemize}
            \item Helps optimise the lookup problem in finding the location of the data
\item Does not require a centralised root server like the dns to enable the lookup 

        \end{itemize}
        \item Disadvantages
        \begin{itemize}
            \item All of the methods use a distributed hash table approach in discovering nodes
\item These algorithms retrieve data based on a unique identifier and keyword based search thereby making the user experience painful
        \end{itemize}
    \end{itemize}
\end{frame}


\begin{frame}{Multi-Torrent: a Performance Study}
    \begin{itemize}
        \item Incentivizing users to stay in a multi-torrent environment and investigates the performance impact due to the same
-------------------------------------------------------------------------------------------------------------------------------------------------------------------------------------------------------------------------------------        \item Advantages
        \begin{itemize}
            \item Multiple torrents allows for faster download speeds
            \item Incentives are provided to the seeders so that they have reason to keep the torrent alive
        \end{itemize}
        \item Disadvantages
        \begin{itemize}
            \item The existing protocol has to be updated, and this might result in a loss of interoperability with other torrent clients
        \end{itemize}
    \end{itemize}
\end{frame}

\begin{frame}{Analyzing The DC File Sharing Network}
    \begin{itemize}
        \item DC (Direct Connect) Network is a P2P file-sharing network, which is the second most popular file sharing network (after BitTorrent) in some places. It consists of regular users (clients) connected to one or more central hubs.
        \item Advantages
        \begin{itemize}
            \item Users can select hubs which are geographically near, to get high speeds.
        \end{itemize}
        \item Disadvantages
        \begin{itemize}
            \item It doesn’t cache search queries.
            \item During a survey it was seen there was a lot of duplicate text queries (about 40\%)
        \end{itemize}
    \end{itemize}
\end{frame}

\begin{frame}{Comparison Table}
    \begin{tabularx}{1.0\textwidth} { 
        | >{\centering\arraybackslash}X
        | >{\centering\arraybackslash}X
        | >{\centering\arraybackslash}X
        | >{\centering\arraybackslash}X | }
        \hline Paper & Techniques & Pros & Cons \\
        \hline P2P Networking: An Information-Sharing Alternative & P2P network as a mode of file sharing over normal client/server model, no central server & Enhanced load balancing, redundancy and fault tolerance, content based addressing, dynamic information repository & Decreased security, copyright infringement, poorer search/retrieval time as nodes may contain only parietal information \\
        \hline PeerDB : A P2P-based System for Distributed Data Sharing & Participating node is a full fledged abject management system supporting content based search  & Good query response time, more fine than coarse file sharing & Not a file transferring applications primarily, works more as a distributed database \\
        \hline
    \end{tabularx}
\end{frame}

\begin{frame}{Comparison Table}
    \begin{tabularx}{1.0\textwidth} { 
        | >{\centering\arraybackslash}X
        | >{\centering\arraybackslash}X
        | >{\centering\arraybackslash}X
        | >{\centering\arraybackslash}X | }
        \hline Paper & Techniques & Pros & Cons \\
        \hline The BitTorrent P2P File-Sharing System: Measurements and Analysis & File downloading based on P2P network & Good average download speed of 240kbps and upwards, system moderators helps maintain integrity of the data & Peers should be incentivised, moderators need a global component not a distributed one, partial fault tolerance is non existent \\
        \hline Looking up data in P2P & Distributed hash table & Does not require a central server & Use a unique identifier and keyword search not supported \\
        \hline
    \end{tabularx}
\end{frame}

\begin{frame}{Comparison Table}
    \begin{tabularx}{1.0\textwidth} { 
        | >{\centering\arraybackslash}X
        | >{\centering\arraybackslash}X
        | >{\centering\arraybackslash}X
        | >{\centering\arraybackslash}X | }
        \hline Paper & Techniques & Pros & Cons \\
        \hline Multi-Torrent: a Performance Study & Multi Torrent seeding & Faster transfer speeds, incentivises users to seed & Slightly updated version of bit torrent protocol hence all clients may not support it \\
        \hline Analyzing The DC File Sharing Network & Hub model & Requires a central hub for distribution & Search queries are not cache, many duplicate text queries were present ``\\
        \hline
    \end{tabularx}
\end{frame}



\begin{frame}
 \large{ THANK YOU !!}
\end{frame}

\end{document}


\section{}
\begin{frame}
 \large {Thank You}
\end{frame}


\end{document}
